\documentclass{standalone}
\usepackage{tikz}

\begin{document}

\tikzset{
    node circle/.style={
        circle,
        draw=black!80,
        very thick,
        inner sep=1pt,
        minimum size=4pt
    },
    edge arrow/.style={
        ->,
        line width=1.2pt,
        draw=black!80,
        shorten >=1pt
    },
    input arrow/.style={
        <-,
        line width=1.5pt,
        draw=red!75!black,
        shorten <=1pt
    },
    output arrow/.style={
        ->,
        line width=1.5pt,
        draw=blue!55!black,
        shorten >=1pt
    }
}

\begin{tikzpicture}

\def\unitheight{1.25}
\def\unitwidth{1.5}

\def\inputangle{225}
\def\inputradius{1}

\def\outputangle{45}
\def\outputradius{1}

\node [node circle, label=145:A] (A) at (0, 0) {};
\node [node circle, label=100:B] (B) at (2 * \unitwidth, 0) {};
\node [node circle, label=100:C] (C) at (4 * \unitwidth, 0) {};
\node [node circle, label=0:D] (D) at (6 * \unitwidth, -1 * \unitheight) {};
\node [node circle, label=170:E] (E) at (1 * \unitwidth, -2 * \unitheight) {};
\node [node circle, label=280:F] (F) at (3 * \unitwidth, -2 * \unitheight) {};

\draw [edge arrow] (A) -- (B);
\draw [edge arrow] (B) -- (C);
\draw [edge arrow] (C) -- (D);
\draw [edge arrow] (E) -- (F);
\draw [edge arrow] (F) -- (D);
\draw [edge arrow] (E) -- (B);
\draw [edge arrow] (B) -- (F);
\draw [edge arrow] (F) -- (C);

\draw[input arrow] (A) -- ++ (\inputangle:\inputradius);
\draw[input arrow] (E) -- ++ (\inputangle:\inputradius) node[left] {entrées};
\draw[input arrow] (F) -- ++ (\inputangle:\inputradius);

\draw[output arrow] (B) -- ++ (\outputangle:\outputradius);
\draw[output arrow] (C) -- ++ (\outputangle:\outputradius) node[right] {sorties};
\draw[output arrow] (D) -- ++ (\outputangle:\outputradius);

\end{tikzpicture}
\end{document}
